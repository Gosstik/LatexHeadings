%%%%%%%%%%%%%%%%%%%%%%%%%%%%%%%%%%%%%%%%%%%%%%%%%%%%%%%%%%%
%%%% Utilities
%%%%%%%%%%%%%%%%%%%%%%%%%%%%%%%%%%%%%%%%%%%%%%%%%%%%%%%%%%%

%%%%%%%%%%%%%%%%%% Simple text %%%%%%%%%%%%%%%%%%

% в тексте << и >> заменяются на кавычки
% из-за пакета \babel кавычки задаются двумя символами:
%% "` -- открывающая кавычка
%% "' -- закрывающая кавычка

% тире: ---
% тире между подлежащим и сказуемым: "---

% символ "~" или "\ " обозначает дополнительный пробел

% \verb|some text| -- prints text with monospaced font ("|" is delimiter in that case, but any character except "*" can be delimiter)

% Посмотреть значение параметра: \the\parskip
%% Подробнее: https://tex.stackexchange.com/questions/693280/displaying-values-of-latex-variables

% \section[short title]{full title}

% \appendix

% \pagestyle{empty} -- сделать страницу без колонтитулов
% \thispagestyle

% \hspace{10pt}, \vspace{10pt}
% \hskip, \vskip
% \\, \newline -- перейти на новую строку, не создавая абзаца
% \linebreak, \pagebreak -- если строки вылезают за границы страницы

% \par == пустая строка -- переход к новому абзацу

% \begin{tabular}{||c|c|||r|l||} -- создание таблицы
% \hline -- горизонтальная черта

% \includegraphics[width=0.8\textwidth]{Images/cat}
%% Объяснение:
%% \textwidth -- константа, встроенная в латех
%% 0.8 -- числовой множитель
%% width= -- размер изображения, если оно слишком большое

% \newcommand{\percent}{\mathbin{\%}} -- сделали бинарный оператор с помощью \mathbin (в таком случае правильно расставляются пробелы вокруг оператора)

% \begin{itemize}, \begin{enumerate} -- нумерованные списки

%%% Удобно использовать в доказательстве:
% \begin{itemize}
%     \item[$\Leftarrow$]
%     \item[$\Rightarrow$]
% \end{itemize}

%%% Document sectioning
% -1 -- \part{part} -- используется в основном в книгах
% 0  -- \chapter{chapter} -- используется в основном в книгах
% 1  -- \section{section} -- используется в статьях
% 2  -- \subsection{subsection}
% 3  -- \subsubsection{subsubsection}
% 4  -- \paragraph{paragraph}
% 5  -- \subparagraph{subparagraph}

%%%%%%%%%%%%%%%%%% Theorems %%%%%%%%%%%%%%%%%%

% \theoremstyle{plain} -- жирное название, курсивные текст
% \theoremstyle{definition} -- жирное название, обычный текст
% \theoremstyle{remark} -- курсивное название, обычный текст

% \newtheorem{theorem}{Теорема}[section] -- в [] указывается счётчик (т.е. нумерация теоремы в section 1 будет 1.1, 1.2 ...)
% \newtheorem{corollary}{Следствие}[theorem] -- в section 1 после теоремы 1 нумерация будет 1.1.1, 1.1.2, ...
% \newtheorem{corollary}[theorem]{Следствие} -- в section 1 у теорем и следствий будет один и тот же счётчик (т.е будет теорема 1.1, следствие 1.2, теорема 1.3, ...)
% \newtheorem*{...}{...} -- без нумерации

% в конце доказательства ставится символ \qedsymbol (по умолчанию - белый квадрат). Этот символ можно  поменять с помощью \renewcommand{\qedsymbol}{new_symbol}

%%%%%%%%%%%%%%%%%% Counters %%%%%%%%%%%%%%%%%%

% \newcounter{name}
% \newcounter{name}{other_name} -- второй счёичик обнуляется каждый раз, когда инкрементируется первый

% Изменение счётчика
%% \stepcounter{name}
%% \addcounter{name}{number}
%% \setcounter{name}{number}

% обращение к счётчику (и вывод значения на экран)
%% Пусть \newcounter{name}, тогда:
%% \arabic{name} -- 0
%% \Roman{name} -- только положительные
%% \alph{name} -- буквы латинского алфавита (от 0 до 26)

% \being{enumerate}[label=\Alph*]
%     \item item A
%     \setcounter{enumi}{5}
%     \item another item
% \end{enumerate}

%%%%%%%%%%%%%%%%%% Environments %%%%%%%%%%%%%%%%%%

% \begin{center} -- центрированный текст
% \begin{equation} -- мат среда с нумерацией
% \begin{equation*} -- мат среда без нумерации (тогда эквивалентно \[\])
% \begin{align} -- можно выбирать, по какому символу ровнять строки. Нумерует каждую строку отдельно
% \begin{align*} -- аналогично без нумерации
% \begin{gather} -- выравнивает всё по центру
% \begin{gather*} -- аналогично без нумерации
% \begin{multiline} -- ровняет строки, чтобы они находились на одной диагонали
% \begin{multiline*} -- аналогично без нумерации
% \begin{aligned} -- для использования внутри математической формулы
% \begin{cases} -- для систем уравнений

% \newenvironment{name}{preceding_code}{succeeding_code}
%% Пример
%% \newenvironment{myenv}
%% {\begin{center}\bfseries}
%% {\end{center}}

%%%%%%%%%%%%%%%%%% Math mode %%%%%%%%%%%%%%%%%%

% Виды шрифтов в математическом окружении
%% \mathbb{} -- для обозначения множеств
%% \mathbf{} -- жирное начертание (для обозначения вектора)
%% \mathrm{} -- для операторов
%% \mathsf{} -- без засечек (для матожа, например)
%% \mathcal{} -- каллиграфический

% \frac{} -- (fraction) внутри строки (т.е. $$) отрисовывает небольшую дробь, в математическом окружении - полноразмерную
% \dfrac{} -- (display fraction) всегда отрисовывает полноразмерную дробь

% \xleftrightarrow[below]{above}

% \dot{a}
% \ddot{a}
% \mathring{a}
% \hat{} -- \widehat{}
% \tilde{} -- \widetilde{}
% \bar{} -- \overline{}
% \vec{} -- \overrightarrow{}

% Можно расставлять свою нумерацию с помощью \tag{}
%% \begin{align*}
%%     F = \frac{1}{2} \tag{III 3H}
%% \end{align*}

% \begin{pmatrix}, {bmatrix}, {vmatrix}...
% NiceMatrix package (\begin{pNiceMatrix}, \begin{pNiceArray}{cc|c})

% \cdots (horizontal)
% \vdots (vertical)
% \ddots (diagonal)

%%% Spacing
% \!      -- -3/18 of \quad (= -3mu)
% \,      -- 3/18 of \quad (= 3mu)
% \:      -- 4/18 of \quad (= 4mu)
% \;      -- 5/18 of \quad (= 5mu)
% \       -- equivalent of space in normal text
% \quad   -- space equal to the current font size (= 18 mu)
% \qquad  -- twice of \quad (= 36 mu)

%%% features
% \underset{text}{under_text}  --- подпись под текстом

%%%%%%%%%%%%%%%%%% References %%%%%%%%%%%%%%%%%%

% \label{eq:3nl} -- название создаваемой ссылки
% \ref{eq:3nl} -- обращение к ссылке, созданной \label{}
% Существуют также:
%% \eqref{}
%% \pageref{} -- чтобы прописывался номер страницы
%% \figref{}
%% \href{site}{name} -- для гиперссылок

% Пример
%% \begin{align*}\label{eq:3nl}
%%     F = \frac{1}{2} \tag{III 3H}
%% \end{align*}

% \footnote{это сноска \label{fn}} -- сноска
%% на сноски можно ссылаться: \ref{fn}

% Сокращение для ссылок в зависимости от объекта, на который ссылаемся:
%% ch: -- chapter
%% sec: --- section
%% subsec: --- subsection
%% fig: --- figure
%% tab: --- table
%% eq: --- equation
%% lst: --- code listing

% Создание ссылки на конерктный текст в документе:
%% \hypertarget{name}{text} -- чтобы создать ссылку
%% \hyperlink{name}{local_text} -- чтобы сослаться
%% Больше про навигацию по документу: https://www.unix-lab.org/posts/hyperref/

%% Цвета ссылок:
% linkcolor -- цвет гиперссылок внутри документа (по-умолчанию red)
% pagecolor -- цвет гиперссылок на другие страницы внутри документа (по-умолчанию red)
% filecolor -- цвет гиперссылок, открывающих локальные файлы (по-умолчанию cyan)
% anchorcolor -- цвет текста мишени (по-умолчанию black)
% citecolor -- цвет библиографических ссылок (по-умолчанию green)
% urlcolor -- цвет гиперссылок на сетевые ресурсы (по-умолчанию magenta)

%%%%%%%%%%%% Подключание файлов %%%%%%%%%%%%%%%%%

% \input{preamble}
%
% \begin{document}
%     \input{titlepage}
%     \input{chapter1}
%     \input{chapter2}
%     \input{chapter3}
% \end{document}

% \input{} принимает относительный путь от главного файла (который компилируется) - это относится и к вложенным \input{}.

% есть также команды \include{} (переходит на новую строку, не может быть вложенным!) и добавление к ней - \includeonly{name1, name2} (чтобы компилировать только часть)

% Пути, указываемые до файлов, всегда смотрятся относительно самого главного файла (компиляция которого запускается)

%%%%%%%%%%%% Изображения %%%%%%%%%%%%%%%%%

% \usepackage{pdfpages} -- для \includepdf[pages={1,3,5-6}]{filename.tex}

%%%% Man of "minipage" environment (for inserting images)
% https://latex-tutorial.com/minipage-latex/

%%%% Можно делать рамку вокруг изображения, передавая дополнительный аргумент:
%% "fbox" -- можно самому настраивать границы и толщину (\fboxsep, \fboxrule)
% \includegraphics[width=\textwidth,fbox]{path-to-image}
%% "frame" -- показывает границы изображения:
% \includegraphics[width=\textwidth,frame]{path-to-image}


%%%%%%%%%%%% tikz %%%%%%%%%%%%%%%%

%%%%%%%%%%%%%%%%%%%%%%%%%%%%%%%%%%

%%%%%%%%%%%%%%%%%%%%%%%%%%%%%%%%%%%%%%%%%%%%%%%%%%%%%%%%%%%
%%%% Оформление страниц
%%%%%%%%%%%%%%%%%%%%%%%%%%%%%%%%%%%%%%%%%%%%%%%%%%%%%%%%%%%

\AddToHook{cmd/section/before}{\clearpage}  % начинать каждую секцию с новой страницы

%%%%%%%%%%%%%%%%%%%%%%%%%%%%%%%%%%%%%%%%%%%%%%%%%%%%%%%%%%%
%%%% Colors.
%%%%%%%%%%%%%%%%%%%%%%%%%%%%%%%%%%%%%%%%%%%%%%%%%%%%%%%%%%%

\definecolor{linkcolor}{HTML}{50006b} % цвет ссылок
%\definecolor{urlcolor}{HTML}{107896} % цвет гиперссылок
\definecolor{urlcolor}{HTML}{50006b} % цвет гиперссылок

\hypersetup{
  unicode=true,            % русские буквы в разделе PDF
  colorlinks=true,       	 % Цветные ссылки вместо ссылок в рамках
  linkcolor=black!10!blue, % Внутренние ссылки
  citecolor=green,         % Ссылки на библиографию
  filecolor=magenta,       % Ссылки на файлы
  urlcolor=cyan,
  pdfstartview=FitH,
}

%%%%%%%%%%%%%%%%%%%%%%%%%%%%%%%%%%%%%%%%%%%%%%%%%%%%%%%%%%%
%%%% Enumerate and itemize
%%%%%%%%%%%%%%%%%%%%%%%%%%%%%%%%%%%%%%%%%%%%%%%%%%%%%%%%%%%

% \linespread{1.3}                    % spaces between lines
\setlength{\parindent}{12pt}        % indentation of first line of the paragraph CHECK itemparindent AFTER CHANGING THAT VALUE
\setlength{\parskip}{5pt}           % space between paragraphs
\newcommand\itemparindent{20pt} % indentation of first line of the paragraph CHECK ENUMERATE AFTER CHANGING THAT VALUE

\setlist{nosep}                     % spaces inside "enumerate" and "itemize"
% \setlist{noitemsep}

%%%%%%%%%%%% Enumerate %%%%%%%%%%%%

\renewcommand{\alph}[1]{\asbuk{#1}} % hack for cyrillic items
\renewcommand{\Alph}[1]{\Asbuk{#1}} % hack for cyrillic items

\newcommand\customlabelsep{5pt}

\setlist[enumerate]{
  wide,
  align=left,
  left=0pt .. \itemparindent, % <labelindent> .. <leftmargin>
  itemindent=*,
  leftmargin=*,
  labelindent=20pt,
  listparindent=0pt,
  labelsep=\customlabelsep,
  itemsep=\parskip,
  parsep=\parskip,
  % topsep=\parskip,
}

\setlist*[enumerate, 1]{
  label=\arabic*),
}
\setlist*[enumerate, 2]{
  label=\arabic{enumi}.\arabic*),
}
\setlist*[enumerate, 3]{
  label=\Alph*),
  % labelwidth=!, % костыль из-за неправильного выравнивания
}

\setlist*[enumerate, 4]{
  label=\Roman*),
  % labelwidth=!, % костыль из-за неправильного выравнивания
}

%%%%%%%%%%%% Itemize %%%%%%%%%%%%

%%% More items: https://latex-tutorial.com/bullet-styles/

\setlist[itemize]{
  wide,
  align=left,
  left=0pt .. \itemparindent, % <labelindent> .. <leftmargin>
  itemindent=*,
  leftmargin=*,
  labelindent=20pt,
  listparindent=0pt,
  labelsep=\customlabelsep,
  itemsep=\parskip,
  parsep=\parskip,
  % topsep=\parskip,
}

\setlist*[itemize, 1]{
  label=\ding{117},
}

\setlist*[itemize, 2]{
  label=$\bullet$,
}

\setlist*[itemize, 3]{
  label=\ding{228},
  % labelwidth=*, % костыль из-за неправильного выравнивания
}

\setlist*[itemize, 4]{
  label=\ding{111},
  % labelwidth=*, % костыль из-за неправильного выравнивания
}

%%%%%%%%%%%%%%%%%%%%%%%%%%%%%%%%%%%%%%%%%%%%%%%%%%%%%%%%%%%
%%%% Code snippets.
%%%%%%%%%%%%%%%%%%%%%%%%%%%%%%%%%%%%%%%%%%%%%%%%%%%%%%%%%%%

% https://www.overleaf.com/learn/latex/Code_listing

%%% Some colors for code.
% mycodestyle
\definecolor{codegreen}{rgb}{0,0.6,0}
\definecolor{codegray}{rgb}{0.5,0.5,0.5}
\definecolor{codepurple}{rgb}{0.58,0,0.82}
% \definecolor{codebackcolour}{rgb}{0.95,0.95,0.92}
\definecolor{codebackcolour}{RGB}{242,242,242}
% cpp
\definecolor{cppbackgroundcolor}{RGB}{255,255,255}
\definecolor{cppidentifiercolor}{HTML}{000000}
\definecolor{cppcommentcolor}{HTML}{57C747}
\definecolor{cppstringcolor}{HTML}{709040}
\definecolor{cppkeywordcolor}{HTML}{0000FF}

%%% Style of code.
% Accessible font families: https://tex.stackexchange.com/questions/378623/how-can-i-get-a-list-of-all-fontfamily-fontseries-combinations/378626#378626
\lstdefinestyle{mycodestyle}{
  basicstyle=\ttfamily\footnotesize,
  % basicstyle=\fontfamily{ccr}\selectfont\footnotesize,  % with bold, but == displays incorrectly
  backgroundcolor=\color{codebackcolour},
  commentstyle=\color{codegreen}\bfseries,
  stringstyle=\color{codepurple},
  % keywordstyle=\color{magenta}\bfseries,
  keywordstyle=\color{blue}\bfseries,
  breakatwhitespace=false,
  breaklines=true,
  captionpos=b,
  keepspaces=true,
  numbers=left, % =none to disable line numeration
  numbersep=5pt,
  numberstyle=\tiny\color{codegray},
  showspaces=false,
  showstringspaces=false,
  showtabs=false,
  tabsize=2,
  frame=single,
  texcl=true, % russian comments
}

\lstdefinestyle{cppcodestyle}{
  style=mycodestyle,
  language=C++,
}

% \lstdefinestyle{cppcodestyle}{
%     language=C++,
%     basicstyle=\fontfamily{cmr}\selectfont\footnotesize\color{cppidentifiercolor},
%     backgroundcolor=\color{white},
%     identifierstyle=\color{cppidentifiercolor},
%     commentstyle=\color{cppcommentcolor}\bfseries,
%     stringstyle=\color{cppstringcolor},
%     keywordstyle=\color{cppkeywordcolor}\bfseries,
%     breakatwhitespace=false,
%     breaklines=true,
%     captionpos=b,
%     keepspaces=true,
%     numbers=left, % =none to disable line numeration
%     numbersep=5pt,
%     numberstyle=\tiny\color{codegray},
%     showspaces=false,
%     showstringspaces=false,
%     showtabs=false,
%     tabsize=2,
%     frame=single,
%     texcl=true, % russian comments
% }

% \lstset{style=mycodestyle}
\lstset{style=cppcodestyle}
\lstset{upquote=true}   % using quotes for strings

%%%%%%%%%%%%%%%%%%%%%%%%%%%%%%%%%%%%%%%%%%%%%%%%%%%%%%%%%%%
%%%% Комментарии и замечания.
%%%%%%%%%%%%%%%%%%%%%%%%%%%%%%%%%%%%%%%%%%%%%%%%%%%%%%%%%%%

%%%%%%%%%% package "todonotes". %%%%%%%%%%

%%% Good documentation is available for package "todonotes".

\newlength{\nopartodostrlen}

\newcommandx{\nopartodo}[2][1=]{
  \settowidth{\nopartodostrlen}{{#2}}
  \setlength{\nopartodostrlen}{\nopartodostrlen + 18pt}
  \todo[list, fancyline, inline, noinlinepar, inlinewidth=\nopartodostrlen]{{#2}}
  \ % space for better view
}

\newcommandx{\partodo}[2][1=]{
  \todo[list, fancyline, inline, inlinepar]{{#2}}
}

\newcommandx{\boxtodo}[2][1=]{\todo[list, fancyline, noinline]{{#2}}}

%%% uncomment to disable comments:
% \renewcommandx{\nopartodo}[2][1=]{\todo[disable, {#1}]{{#2}}}
% \renewcommandx{\partodo}[2][1=]{\todo[disable, {#1}]{{#2}}}
% \renewcommandx{\boxtodo}[2][1=]{\todo[disable, {#1}]{{#2}}}

%%% Another parameters (more in docs)
% caption={Caption text} -- name of todo in the list of todos
% prepend/noprepend -- whether to add caption in the todo text
% inlinewidth=5cm -- width of box
% disable -- disable comment

%%%%%%%%%% colorised comment %%%%%%%%%%

\definecolor{commentcolor}{RGB}{255, 136, 0}
% \def\comment#1{{\color{myorange}#1}}
\renewcommand{\comment}[1]{
  {\bfseries\color{commentcolor}{#1}}
}

%%%% For draft version.
\ifdraft
\else
  \renewcommand{\comment}[1]{}
  \renewcommandx{\nopartodo}[2][1=]{\todo[disable, {#1}]{{#2}}}
  \renewcommandx{\partodo}[2][1=]{\todo[disable, {#1}]{{#2}}}
  \renewcommandx{\boxtodo}[2][1=]{\todo[disable, {#1}]{{#2}}}
\fi


%%%%%%%%%%%%%%%%%%%%%%%%%%%%%%%%%%%%%%%%%%%%%%%%%%%%%%%%%%%
%%%% Dates
%%%%%%%%%%%%%%%%%%%%%%%%%%%%%%%%%%%%%%%%%%%%%%%%%%%%%%%%%%%

%%%%%%%%%% Last modified time %%%%%%%%%%

% https://tex.stackexchange.com/questions/137184/how-to-specify-a-date-and-then-use-it-with-a-time-format-defined-using-the-packa

\newdateformat{lastUpdatedDateFormat}{\monthnameenglish[\THEMONTH] \THEYEAR}
% \newdateformat{lastUpdatedDateFormat}{\monthnameenglish[\THEMONTH]\ \THEDAY,\ \THEYEAR}

\newcommand{\displayLastUpdatedDate}[1]{
  \textit{Last updated in \lastUpdatedDateFormat\displaydate{#1}}
}

\newcommand{\displayAutoLastUpdatedDate}{
  \textit{Last updated in \lastUpdatedDateFormat\today}
}

\newcommand{\displayUpperRight}[1]{% \placetextbox{<horizontal pos>}{<vertical pos>}{<stuff>}
  \AddToShipoutPictureFG*{% Add <stuff> to current page foreground
    \put(
    \LenToUnit{\paperwidth-2 cm-0.2 cm+0.05cm},
    \LenToUnit{\paperheight-1.0 cm}
    ){\vtop{{\null}\makebox[0pt][c]{
          \small\color{gray}{#1}
          \hspace{\widthof{#1}}}
      }}}%
}%


%%%%%%%%%%%%%%%%%%%%%%%%%%%%%%%%%%%%%%%%%%%%%%%%%%%%%%%%%%%
%%%% Tikzpicture
%%%%%%%%%%%%%%%%%%%%%%%%%%%%%%%%%%%%%%%%%%%%%%%%%%%%%%%%%%%

%%%%%%%%%% Example %%%%%%%%%%

%-------------%
% \begin{tikzpicture}[node distance=2cm]

% \node (ClientApp) [rect] {\shortstack{Client app \\ \\ frontend}};
% \node (Server) [rect, right of=ClientApp, xshift=5cm] {Your phisical or virtual server};

% \draw [arrow] (ClientApp) -- (Server);

% \end{tikzpicture}
%------------%

%%%%%%%%%% Onine graphical editor %%%%%%%%%%

% tikzmaker: https://tikzmaker.com/editor
% stackoverflow: https://tex.stackexchange.com/questions/84890/is-there-an-online-diagram-graphical-editor-that-produces-the-corresponding-late
